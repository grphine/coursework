

\section{Introduction}

The logic required for the comparator can be determined using a truth table. There are three expected outputs for 4-bit inputs `A' and `B':
\begin{itemize}
    \item A > B (Greater)
    \item A = B (Equal)
    \item A < B (Lesser)
\end{itemize}

\begin{xltabular}{\linewidth}{|YYYY|YYY|}
    \hline
    \rowcolor{tableh1}
    \multicolumn{4}{|c}{Input} & \multicolumn{3}{|c|}{Output} \\
    \hline
    \rowcolor{tableh2}
    A\textsubscript{3}, B\textsubscript{3} & A\textsubscript{2}, B\textsubscript{2} & A\textsubscript{1}, B\textsubscript{1} & A\textsubscript{0}, B\textsubscript{0} & A > B (G) & A = B (E) & A < B (L) \\
    \hline
    A\textsubscript{3} > B\textsubscript{3} & X & X & X & 1 & 0 & 0 \\
    A\textsubscript{3} = B\textsubscript{3} & A\textsubscript{2} > B\textsubscript{2} & X & X & 1 & 0 & 0 \\
    A\textsubscript{3} = B\textsubscript{3} & A\textsubscript{2} = B\textsubscript{2} & A\textsubscript{1} > B\textsubscript{1} & X & 1 & 0 & 0 \\
    A\textsubscript{3} = B\textsubscript{3} & A\textsubscript{2} = B\textsubscript{2} & A\textsubscript{1} = B\textsubscript{1} & A\textsubscript{0} > B\textsubscript{0} & 1 & 0 & 0 \\
    \hline
    A\textsubscript{3} = B\textsubscript{3} & A\textsubscript{2} = B\textsubscript{2} & A\textsubscript{1} = B\textsubscript{1} & A\textsubscript{0} = B\textsubscript{0} & 0 & 1 & 0 \\
    \hline
    A\textsubscript{3} < B\textsubscript{3} & X & X & X & 0 & 0 & 1 \\
    A\textsubscript{3} = B\textsubscript{3} & A\textsubscript{2} < B\textsubscript{2} & X & X & 0 & 0 & 1 \\
    A\textsubscript{3} = B\textsubscript{3} & A\textsubscript{2} = B\textsubscript{2} & A\textsubscript{1} < B\textsubscript{1} & X & 0 & 0 & 1 \\
    A\textsubscript{3} = B\textsubscript{3} & A\textsubscript{2} = B\textsubscript{2} & A\textsubscript{1} = B\textsubscript{1} & A\textsubscript{0} < B\textsubscript{0} & 0 & 0 & 1 \\
    \hline
    \caption{Comparator truth table}\label{table:truth}
\end{xltabular}

To simplify the set of equations, the truth table output can be modified such that only G and E need to be considered. L can then be inferred by whether G and E are both 0.
This gives the resultant equations:
\vspace{11pt}

\begin{lstlisting}[label=boolequg, caption={`A > B' equations},captionpos=b]
G = A3B3' + A3A2B2' + A2B3'B2' + A3A2A1B1' + A2A1B3'B1' + A3A1B2'B1' + A1B3'B2'B1' + A3A2A1A0B0' + A2A1A0B3'B0' + A3A1A0B2'B0' + A1A0B3'B2'B0' + A3A2A0B1'B0' + A2A0B3'B1'B0' + A3A0B2'B1'B0' + A0B3'B2'B1'B0'
\end{lstlisting}

\begin{lstlisting}[label=booleque, caption={`A = B' equations},captionpos=b]
E = A3A2A1A0B3B2B1B0 + A3'A2A1A0B3'B2B1B0 + A3A2'A1A0B3B2'B1B0 + A3'A2'A1A0B3'B2'B1B0 + A3A2A1'A0B3B2B1'B0 + A3'A2A1'A0B3'B2B1'B0 + A3A2'A1'A0B3B2'B1'B0 + A3'A2'A1'A0B3'B2'B1'B0 + A3A2A1A0'B3B2B1B0' + A3'A2A1A0'B3'B2B1B0' + A3A2'A1A0'B3B2'B1B0' + A3'A2'A1A0'B3'B2'B1B0' + A3A2A1'A0'B3B2B1'B0' + A3'A2A1'A0'B3'B2B1'B0' + A3A2'A1'A0'B3B2'B1'B0' + A3'A2'A1'A0'B3'B2'B1'B0'
\end{lstlisting}


\begin{wraptable}[5]{R}{0.2\textwidth}
    \vspace{-33pt}
    \centering
    \begin{xltabular}{\linewidth}{YY|Y}
        \hline
        \rowcolor{tableh1}
        G & E & \cellcolor{tableh2}L \\
        \hline
        0 & 0 & 1 \\
        X & 1 & 0 \\
        1 & X & 0 \\
        \hline
        \caption{L truth table}\label{table:truth2}
    \end{xltabular}
\end{wraptable}
A schematic can then be designed. To accommodate L, an additional gate is required to evaluate the output of G and E.
G and E should not be able to assert at the same time\footnote{
    It may be important to ensure outputs cannot assert at the same time.
}, however for the calculation of L it is irrelevant.
Therefore, either a NOR or an XNOR gate is appropriate. The choice ultimately falls as to which would be most appropriate
for the given design.
\begin{wraptable}[9]{L}{0.28\textwidth}
    \vspace{-5pt}
    % \centering
\begin{tabular}{ccc}
    \hline
    \rowcolor{tableh1}
    Gate                & Inputs & Quantity \\
    \hline
    NOT                 & 1      & 9        \\
    \hline
    \multirow{2}*{NAND} & 2      & 10       \\
                        & 3      & 3        \\
    \hline
    \multirow{2}*{NOR}  & 2      & 1        \\
                        & 3      & 1        \\
    \hline
\end{tabular}
\caption{Gate usage} \label{table:gates}
\end{wraptable}

% \begin{itemize}
%     \item NOT x9
%     \item 2-input NAND x10
%     \item 3-input NAND x3
%     \item 2-input NOR x1
%     \item 3-input NOR x1
% \end{itemize}

In total, four gate types are used for the schematic in \cref{fig:schematic}, shown in \cref{table:gates},
which overall required 86 transistors. However, it 
is possible to reduce the number of transistors or the number of gate ICs.
During research and design, the fewest transistor count determined possible
was 70. As this requires a 5-input NAND gate\footnote{
    Gates with 4 series transistors begin to suffer delays such that it is better to split the logic across multiple gates.
}, the next most feasible design uses 80 transistors. Both of these designs are reliant on using a non-standard XOR gate
in order to determine equality between A and B. While typically an XOR gate requires 8 transistors,
the fewest possible is 3 \cite{chandra2015}, however the output suffers from signal degradation and requires a
buffer to clean the output. While this method may reduce the overall die area required,
it adds a layer of complexity to both the overall design process, and per component simulation.

\begin{figure}[H]
    
    \centering
    \scalebox{0.8}{
        \begin{tikzpicture}
            \ctikzset{
                logic ports=ieee,
                % logic ports/scale=1,
                logic ports/fill=lightgray
            }
            % gates
            \node[not port](a31) at (0,0){};
            \node[nand port](a32) at (3.5,-0.28){};
            \node[nand port, number inputs=3](a33) at (6.75,0.1){};

            \node[not port](a21) at (0,-1.75){};
            \node[nand port](a22) at (3.5,-2.03){};
            \node[nand port](a23) at (6.75,-1.75){};
            \node[not port](a24) at (9,-1.75){};
            \node[nand port](a25) at (11.25,-2.03){};

            \node[not port](a11) at (0,-3.5){};
            \node[nand port](a12) at (3.5,-3.78){};
            \node[nand port, number inputs=3](a13) at (6.75,-3.41){};

            \node[not port](a01) at (0,-5.25){};
            \node[nand port](a02) at (3.5,-5.53){};
            \node[nand port](a03) at (6.75,-5.25){};
            \node[nor port, number inputs=3](a04) at (16,-5.25){X2};

            \node[not port](b31) at (0,-7){};
            \node[nand port](b32) at (3.5,-7.28){};
            \node[nand port, number inputs=3](b33) at (13.75,-6.9){X1};
            \node[nor port](b34) at (18.5,-6.62){X3};

            \node[not port](b21) at (0,-8.75){};

            \node[not port](b11) at (0,-10.5){};
            \node[nand port](b12) at (3.5,-10.78){};
            \node[nand port](b13) at (9,-10.5){};

            \node[not port](b01) at (0,-12.25){};

            % inputs
            \draw (a31.in) -- ++(-0.5,0)node[left](A3){A\textsubscript{3}};
            \draw (a21.in) -- ++(-0.5,0)node[left](A2){A\textsubscript{2}};
            \draw (a11.in) -- ++(-0.5,0)node[left](A1){A\textsubscript{1}};
            \draw (a01.in) -- ++(-0.5,0)node[left](A0){A\textsubscript{0}};
            \draw (b31.in) -- ++(-0.5,0)node[left](B3){B\textsubscript{3}};
            \draw (b21.in) -- ++(-0.5,0)node[left](B2){B\textsubscript{2}};
            \draw (b11.in) -- ++(-0.5,0)node[left](B1){B\textsubscript{1}};
            \draw (b01.in) -- ++(-0.5,0)node[left](B0){B\textsubscript{0}};

            %outputs
            \draw (a04.out) -- ++(2.5,0)node[right](E){E};
            \draw (b34.out) -- ++(0,0)node[right](L){L};
            \draw (a04.out) to[short, *-] ++(0,0) |- (b34.in 1);
            \draw (b33.out) to[short, *-] ++(0,0) -- ++(0,-1) -- ++(4.75,0)node[right](G){G};

            %connections    
            \draw (a31.in) to[short, *-] ++(0,-0.75) -- ++(2.25,0) |- (b32.in 2);
            \draw (a31.out) -- (a32.in 1);
            \draw (a32.out) -- (a33.in 3);
            \draw (a33.in 3) to[short, *-] ++(0,0) -- (a23.in 1);
            \draw (a33.out) -- ++(4.75,0) |- (b33.in 1);

            \draw (a21.in) to[short, *-] ++(0,0.75) -- ++(6,0) |- (a33.in 2);
            \draw (a21.out) -- (a22.in 1);
            \draw (a22.out) -- (a23.in 2);
            \draw (a23.out) -- (a24.in);
            \draw (a23.out) ++(0.25,0) to[short, *-] ++(0,0) |- (a04.in 1);
            \draw (a24.out) -- (a25.in 1);
            \draw (a25.out) |- (b33.in 2);

            \draw (a11.in) to[short, *-] ++(0,-0.75) -- ++(2,0) |- (b12.in 2);
            \draw (a11.out) -- (a12.in 1);
            \draw (a12.out) -- (a13.in 3);
            \draw (a13.in 3) to[short, *-] ++(0,0) |- (a03.in 1);
            \draw (a13.out) |- (b13.in 1);

            \draw (a01.in) to[short, *-] ++(0,0.75) -- ++(6.25,0) |- (a13.in 2);
            \draw (a01.out) -- (a02.in 1);
            \draw (a02.out) -- (a03.in 2);
            \draw (a03.out) -- (a04.in 2);

            \draw (b31.in) to[short, *-] ++(0,1) -- ++(2.5,0) |- (a32.in 2);
            \draw (b31.out) -- (b32.in 1);
            \draw (b32.out) -- (b33.in 3);
            \draw (b33.out) |- (a04.in 3);
            \draw (b33.out) to[short, *-] ++(0,0) -- (b34.in 2);

            \draw (b21.in) to[short, *-] ++(0,1) -- ++(2.75,0) |- (a22.in 2);
            \draw (b21.out) -- ++(4,0) |- (a33.in 1);

            \draw (b11.in) to[short, *-] ++(0,1) -- ++(3,0) |- (a12.in 2);
            \draw (b11.out) -- (b12.in 1);
            \draw (b12.out) -- (b13.in 2);
            \draw (b13.out) |- (a25.in 2);

            \draw (b01.in) to[short, *-] ++(0,1) -- ++(3.25,0) |- (a02.in 2);
            \draw (b01.out) -- ++(4.25,0) |- (a13.in 1);
        \end{tikzpicture}
    }
    \caption{Comparator schematic}\label{fig:schematic}
\end{figure}

Another notable feature is the use of NOR gates, especially a 3-input NOR. Typically, NOR gates are avoided
with a preference for NAND gates. The output circuit following gate X1 in \cref{fig:schematic}
can be redrawn in a number of ways as shown in \cref{fig:circuitredraw}.

\begin{figure}[H]      
    \vskip1\baselineskip
    \centering
    \begin{subfigure}{0.35\textwidth}
        \centering
        \scalebox{1.2}{
            \begin{tikzpicture}
                \ctikzset{
                    logic ports=ieee,
                    logic ports/scale=0.8,
                    logic ports/fill=lightgray
                }
                \draw
                (0,0) node[nor port, number inputs=3] (n1) {X2}
                (2,-0.75) node[nor port] (n2) {X3}

                (n1.out) -| node[circ,midway](p1){} (n2.in 1)

                (n1.in 1) -- ++(-0.5,0)node[left]{A}
                (n1.in 2) -- ++(-0.5,0)node[left]{B}

                (p1) -- ++(1.75,0)node[right]{E}
                (n2.out) node[right]{L}
                (-1.3,-1.5) node[left](C){X1} -- (2.9,-1.5) node[right]{G}
                (1.15,-1.5) node[circ]{} |- (n2.in 2)

                (-1,-1.5) node[circ]{} |- (n1.in 3)
                ;
            \end{tikzpicture}
        }
        \caption{NOR-3 circuit}\label{fig:nor3}
    \end{subfigure}
    \hfill
    % \hspace{3cm}    
    \begin{subfigure}{0.6\textwidth}
        \centering
        \scalebox{1.2}{
            \begin{tikzpicture}
                \ctikzset{
                    logic ports=ieee,
                    logic ports/scale=0.8,
                    logic ports/fill=lightgray
                } \draw
                (0,-0.28) node[nor port] (n1) {X2.1}
                (2,-0.5) node[nor port] (n2) {X2.2}
                (4,-1.25) node[nor port] (n3) {X3}

                (n1.out) |- (n2.in 1)
                (n2.out) -| node[circ,midway](p1){} (n3.in 1)
                (n3.out) node[right]{E}
                (p1) -- ++(1.75,0)node[right]{L}

                (n1.in 1) -- ++(-0.5,0)node[left]{A}
                (n1.in 2) -- ++(-0.5,0)node[left]{B}

                (-1.3,-2) node[left]{X1} -- (4.9,-2) node[right]{G}
                (0.9,-2) node[circ]{} |- (n2.in 2)
                (3.15,-2) node[circ]{} |- (n3.in 2)
                ;
            \end{tikzpicture}
        }
        \caption{NOR-2 circuit}\label{fig:nor2}
    \end{subfigure}    
    % \vfill
    \vskip\baselineskip
    \begin{subfigure}{\textwidth}\centering
        % \vspace{11pt}
        \scalebox{1.2}{
            \begin{tikzpicture}
                \ctikzset{
                    logic ports=ieee,
                    logic ports/scale=0.8,
                    logic ports/fill=lightgray
                } \draw
                (0,-0.2) node[not port] (not1) {}
                (0,-1.3) node[not port] (not2) {}
                (2,-0.75) node[nand port] (nand1) {}

                (1,-2.5) node[not port](not3) {}

                (4,-2.5) node[nand port](nand2) {}
                (6,-0.75) node[nand port](nand3) {}
                (6,-2.5) node[not port](not4) {}
                (8,-0.75) node[not port](not5) {}

                (not1.out) |- (nand1.in 1)
                (not2.out) |- (nand1.in 2)
                (nand1.out) |- (nand2.in 1)
                (nand3.out) -- (not5.in)
                (nand2.out) -- node[circ](p1){} (not4.in)
                (p1) |- (nand3.in 2)
                (not3.out) node[circ](p2){} |- (nand2.in 2)
                (p2) |- ++(3,1) |- (nand3.in 1)

                (not1.in) -- ++(-0.5,0) node[left]{A}
                (not2.in) -- ++(-0.5,0) node[left]{B}
                (-1.2,-3.25) node[left]{X1} -- (9,-3.25) node[right]{G}
                (-0.75,-3.25) node[circ]{} |- (not3.in)
                (not5.out) -- (9,-0.75)node[right]{E}
                (not4.out) -- (9,-2.5)node[right]{L}
                ;
            \end{tikzpicture}
        }
        \caption{NAND circuit}
        \label{fig:nand}
    \end{subfigure}
    \caption{Output circuit possibilities}\label{fig:circuitredraw}
    \vfill
\end{figure}
\clearpage
\begin{wraptable}[9]{r}{0.3\textwidth}
    \vspace{-11pt}
    \centering
    % \renewcommand{\arraystretch}{1.35}
    \begin{NiceTabular}{c|cc}[cell-space-limits=3pt]
                              & \multicolumn{2}{c}{\cellcolor{tableh1}Path effort}                       \\
        \hline
        \rowcolor{tableh1}
        Circuit               & g\textsubscript{E}                                 & g\textsubscript{L}  \\
        \hline
        NOR-3, \ref{fig:nor3} & $12 \dfrac{19}{27}$                                & $6 \dfrac{2}{3}$    \\
        \hline
        NOR-2, \ref{fig:nor2} & $138 \dfrac{8}{9}$                                 & $18 \dfrac{14}{27}$ \\
        \hline
        NAND, \ref{fig:nand}  & $120$                                              & $50 \dfrac{22}{27}$ \\
        \hline
    \end{NiceTabular}
    \caption{Logical effort}\label{table:logicaleffort}
\end{wraptable}
A brief study on the logical effort (\cref{table:logicaleffort}) of each circuit indicates how,
as the number of paths grow in
the alternative circuits, the effort required escalates rapidly.
This is due to how path logical effort,\\ $ G = \prod g_i $,
which quickly grows out of control as branching increases. A NAND-only solution also suffers from a greater number
of gate ICs required for the same logic.

\subsection{Delay}
The overall delay of the comparator can be determined by analysing one of the longest paths. 

\begin{figure}[H]
    % \vspace{11pt}
    \scalebox{1}{
        \begin{tikzpicture}
            \ctikzset{
                logic ports=ieee,
                logic ports/scale=0.8,
                logic ports/fill=lightgray
            } \draw

            (0,0) node[not port] (not1) {}
            (0.18,-1.5) node[nand port] (nand0) {}
            (2,-0.225) node[nand port] (nand1) {}
            (4,0.07) node[nand port, number inputs=3] (nand2) {}
            (4,-1.5) node[nand port] (nand3) {}
            (6,-0.155) node[nand port] (nand4) {}
            (8,0.07) node[nand port] (nand5) {}
            (10,-0.155) node[nand port] (nand6) {}
            (12,-0.45) node[nor port, number inputs=3] (nor1) {}
            (14,-1.2) node[nor port] (nor2) {}

            (not1.out) -- (nand1.in 1) 
            (nand1.out) -- (nand2.in 3)
            (nand2.out) -- (nand4.in 1)
            (nand4.out) -- (nand5.in 2) 
            (nand5.out) -- (nand6.in 1)
            (nand6.out) -- (nor1.in 1)
            % (nor1.out) -- (nor2.in 1) 

            (not1.in) -- ++(-0.5,0) node[left]{A1}
            (not1.in) to[short, *-] ++(0,0) |- (nand0.in 1)

            (nand2.in 3) to[short, *-] ++(0,0) -- (nand3.in 1)
            (nand3.out) -| (nor1.in 3)

            (nor1.out) -| node[circ,midway](p1){} (nor2.in 1)
            (p1) -- ++(1.7,0)node[right]{E}
            (nor2.out) node[right]{L}
            (nor1.in 3)  -- ++(0,-1.5) node[circ, midway](p2){} -- ++(3.8,0) node[right]{G}
            (nor2.in 2) -- ++(0,-0.82) node[circ]{}
            ;
        \end{tikzpicture}
    }
    \caption{Longest path}
    \label{fig:long}
\end{figure}

The minimum delay, $D$, of the longest path can be calculated by $N(G \cdot B \cdot H)^{\frac{1}{N}} + P$,

\begin{conditions}
    N & = & Number of gates \\
    G & = & Path logical effort, $\prod g_i$ \\
    B & = & Branching effort, $\prod \frac{C_{on-path} + C_{off-path}}{C_{on-path}}$ \\
    H & = & Electrical effort, $\frac{C_{out}}{C_{in}}$ \\
    P & = & Parasitic delay, $\sum p_i$ \\
\end{conditions}
In this instance,
\begin{align*}
    N & = 8 \\
    G & = 1 \times \left(\frac{4}{3}\right) ^4 \times \left(\frac{5}{3}\right)^2 \times \frac{7}{3} = \frac{44800}{2187}\\
    B & = \frac{3+4}{3} \times \frac{5+4}{5} \times \frac{7+5}{7} = \frac{98}{15}\\
    H & = \frac{6}{3} = 2\\
    P & = 1 + 2 + 3 + 2 + 2 + 2 + 3 + 2 = 17 \\
    \therefore \\
    D & = 33.089
\end{align*}

The absolute delay, $D_{abs} = D\cdot\tau$, where $\tau$ is the process dependent time constant. Assuming 
$\tau \approxeq \qty{3}{\ps}$, $D_{abs} = \qty{99.27}{\ps}$.

With all things considered, the proposed design in \cref{fig:schematic} offers a good balance between size,
propagation delay, and overall complexity.