\clearpage
\section{Evaluation}

\begin{xltabular}{\textwidth}{YYYYYY}
    \hline
    Peak power & Area efficiency & Chip area & Transistor count & Delay & Recommended clock \\
    \hline
    \qty{68.189}{\uW} & \qty{102.233}{\percent} & \qty{19.365}{\um\squared} & 86 & \qty{400}{\ps} & $\leqslant$\qty{2.5}{\GHz}  \\
    \hline    
    \caption{Performance metrics}
\end{xltabular} 

The comparator as a whole appears to function quite well. 
It responds appropriately and quickly to the input of two 4-bit numbers, A and B, to set 
the outputs G (greater), E (equal), or L (lesser). 
Die area is low and utilisation is high, achieved by overlapping standard cells.
The design uses a reasonably low number of transistors, at 86, spread across five 
gate ICs. While designs using fewer transistors are possible, they may introduce 
complexity when maintaining standardised cells or perform slower. 

\subsection{Standard cells}
The design leverages the use of standard cells to attain the extremely high area
efficiency. The cells share a common n-well, implants, and power rails, allowing 
gates to be overlapped. While each individual gate's area efficiency can be improved, 
the aim for standardisation allows the top cell to recoup those losses. The large diffusions 
required may also increase the expense of manufacture. 
The use of taller cells may allow for this expense to be reduced, however that would 
have to balanced against the total area used. 

A downside of designing in this manner is that it leads to routing with little to no 
flexibility to accommodate changes. Using the comparator as a standard cell itself 
may require adjustment of sizing, which may prove difficult. 

\subsection{Routing}
The pins and rails for each cell are exposed on metal 3. Routing on the top cell
then uses metal 4 to move horizontally between cells. As the rails are on 
the same metal layer as the vertical routing, the maximum number of horizontal lines that 
can be supported is reduced. The comparator falls within the level of complexity that 
can be supported, however it is close to the limit of what is possible with a single line
layout given the maximum number of horizontal tracks (six) is reached. 

Routing can be improved by exposing the rails on a different layer to the internal pins (thereby 
allowing overlapping), and increasing the standard cell height for more complex designs. For particularly 
complex designs, routing may not be possible across the cells, in which case pins may need to move 
to be commonly accessible from a routing channel.

\subsection{Power dissipation}
Compact routing across the cells additionally lead to a low value of power dissipation during 
switching. Due to the level of complexity of the design, and thereby the total number of transistors used,
the static power dissipation is in the region of yW and negligible when analysing the comparator by itself.  

\subsection{Delay}
The minimum delay required was calculated to be \qty{100}{\ps}, which is 3.5 times smaller 
than the observed minimum delay, \qty{350}{\ps}. This indicates that there may be areas for 
improvement in the design to minimise this, although managing this without sacrificing on 
other aspects like area utilisation may be difficult to balance. 
One such improvement may be to change the NOR gate across G and E for an XNOR, potentially allowing 
a quicker response by limiting transient outputs. 



