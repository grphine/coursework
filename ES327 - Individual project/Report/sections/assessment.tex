

\section{Assessment}
The project success will be assessed against specified the aims\footnote{
    \Cref{sec:aim}.
} and objectives\footnote{
    \Cref{sec:objectives}.
}.

\subsection{Assessment of objectives}
The objectives outline the steps of the project that needed to be followed for success.
Analysis of met objectives is available in \cref{table:objeval}, where success of a give point is
determined by whether the objective was met \fcolorbox{black}{tableg}{\rule{0pt}{5pt}\rule{5pt}{0pt}}\,, 
unmet \fcolorbox{black}{tabler}{\rule{0pt}{5pt}\rule{5pt}{0pt}}\,, 
or somewhat met \fcolorbox{black}{tablea}{\rule{0pt}{5pt}\rule{5pt}{0pt}}\,. 

{\small
\begin{xltabular}{\linewidth}{|>{\raggedright\arraybackslash}p{6cm}|>{\color{white}}Z|}
    \hline
    \rowcolor{tableh2}
    Objective & \textcolor{black}{Met?} \\
    \hline
    \endhead
    \endfoot

    \hline
    Research cat characteristics & \cellcolor{tableg} \\ \hline
    Research \gls{lora} radio requirements & \cellcolor{tableg} \\ \hline
    Design the system and specification & \cellcolor{tableg} \\ \hline
    Develop the tracker according to specification & \cellcolor{tablea}Further details can be found in \cref{table:speceval}. \\ \hline
    Perform verification tests & \cellcolor{tablea}Tests were performed, but were limited in completeness. More were required. \\ \hline
    Develop \acrshort{pcb}, fit to harness, and live-test & \cellcolor{tabler} \\ \hline
    Certification & \cellcolor{tabler} \\ \hline
    
    \caption{Objectives evaluation}\label{table:objeval}
\end{xltabular}
\vspace{-11pt}
}

Colour progression indicates that,
while the project started off well, completion began to suffer as time passed. 

\subsection{Assessment of aims and specification}

{\small
\begin{xltabular}{\linewidth}{|>{\raggedright\arraybackslash}p{4.5cm}|>{\color{white}}Z|}
    \hline
    \rowcolor{tableh2}
    Aim & \textcolor{black}{Met?} \\
    \hline
    \endhead
    \endfoot

    \rowcolor{tableh1}
    \multicolumn{2}{|c|}{Fundamentals} \\
    \hline
    Obtain \acrshort{gps} location & \cellcolor{tableg} \\ \hline
    Transmit via \gls{lora} & \cellcolor{tableg} \\ \hline
    Transmit remotely for a significant period of time & \cellcolor{tableg}Up to fifteen hours \\ \hline
    Transmit at a regular interval & \cellcolor{tableg}Every five seconds \\ \hline
    

    \rowcolor{tableh1}
    \multicolumn{2}{|c|}{Improvements} \\
    \hline
    Ergonomics & \cellcolor{tabler}The product is not designed for use with a pet \\ \hline
    Stability & \cellcolor{tablea}Only the receiver handles errors\\ \hline
    Installable & \cellcolor{tablea}No specific tools required for installation, however 
        it does not have appropriate casing to be mounted \\ \hline
    Location information easily viewable & \cellcolor{tablea}Information is viewable, but requires familiarity 
        with Bash \\ \hline
    

    \rowcolor{tableh1}
    \multicolumn{2}{|c|}{Production} \\
    \hline
    General safety and suitability & \cellcolor{tabler}No safety measures in place \\ \hline
    Resistance to damage and weather & \cellcolor{tablea}Some resistance to damage \\ \hline
    Certification requirements and law compliance & \cellcolor{tabler} \\ \hline
    Scale considerations & \cellcolor{tabler} \\ \hline
    
    \caption{Aims evaluation}\label{table:aimseval}
\end{xltabular}
\vspace{-11pt}
}

\Cref{table:aimseval} shows that the Fundamental aims have been met, and shows a 
similar colour progression to \cref{table:objeval} in that primary points were met, but 
later ones were not.
By meeting the Fundamental aims, the viability of the product has been demonstrated, 
and much from thereon is to improve and refine the design for consumer suitability.

\subsubsection{Assessment of specification}
This will provide a more granular overview of which elements of the specification
in \cref{sec:spec} were met. 

{\small
\begin{xltabular}{\linewidth}{|>{\raggedright\arraybackslash}p{4.5cm}|>{\color{white}}Z|}
    \hline
    \rowcolor{tableh2}
    Specification & \textcolor{black}{Met?} \\
    \hline
    \endhead
    \endfoot

    \rowcolor{tableh1}
    \multicolumn{2}{|c|}{Reporting} \\
    \hline
    Timing & \cellcolor{tablea}Performs better than \qty{30}{\s}, but no way 
        of minimising transmission gaps to under \qty{300}{\s} \\ \hline
    Distance & \cellcolor{tablea}Subject to more thorough testing to find the reliable limit. 
        Current data shows transmission is possible from over \qty{400}{\m}, however this is not consistent \\ \hline    

    \rowcolor{tableh1}
    \multicolumn{2}{|c|}{Portability} \\
    \hline
    Small form-factor & \cellcolor{tablea}Dimensions measure \qtyproduct{52 x 68 x 17}{\mm}\footnote{\Cref{fig:case2dwg}.} \\ \hline
    Lightweight & \cellcolor{tableg}Weight is \qty{55}{\g}, a \qty{63}{\percent} improvement\\ \hline
    Battery powered & \cellcolor{tableg}Battery life is \qty{15}{\hour}, a \qty{67}{\percent} improvement \\ \hline
    Attachable & \cellcolor{tabler} \\ \hline    

    \rowcolor{tableh1}
    \multicolumn{2}{|c|}{Suitability} \\
    \hline
    Installable & \cellcolor{tabler} \\ \hline
    Compliance & \cellcolor{tabler}  \\ \hline

    \rowcolor{tableh1}
    \multicolumn{2}{|c|}{Reliability} \\
    \hline
    Stable & \cellcolor{tablea}Fault handling in receiver code \\ \hline
    Fault resistant & \cellcolor{tablea}Some shock resistance in the casing \\ \hline
    Repairable & \cellcolor{tableg}No proprietary parts or complex tools are used,
         however technical knowledge is required \\ \hline
    Safety & \cellcolor{tabler} \\ \hline
    Deduplication & \cellcolor{tabler} \\ \hline
    
    \caption{Specification evaluation}\label{table:speceval}
\end{xltabular}
\vspace{-11pt}
}

The greater detailed view in \cref{table:speceval} indicates some points met in the aim may 
not quite meet the specification, with many such reasons overlapping.

\section{Improvements}
Using this project's findings as a basis, there are many points of 
specific improvements that can be explored in future projects. 
These will be broken down into two types: improvements that resolve an
issue that was faced during the project, and improvements that
will increase the quality of the product.

\subsection{Issue resolvers}
\paragraph{Battery connection}
There was no way to control the power to the transmitter. As soon as the
battery was connected, it would begin to transmit, and more crucially, drain
the battery. Repeated disconnections of the power port led to the
cables exposing and almost shorting at one point.
This was resolved using insulting tape\footnote{
    The taping can be seen in \cref{fig:case1img}.
}, however such a
method is only a temporary measure.

A better way of managing this
would be to use a switch across one of the battery lines,
allowing power to be killed gracefully. This has a drawback
itself in that it would require the user to power on the device
in order to recharge the battery.
A smarter, user-friendly approach would require discrete charging
circuitry to handle charging and power, which is moving towards
bespoke circuit design and out of the scope of this project. However,
this would be vital towards making this a viable consumer product.

\paragraph{Transmission errors}
Transmission errors only appeared to occur while the system was
used on a close-range basis. However, it is entirely possible for
such errors to occur on a general basis. To minimise this,
a \acrshort{crc} may be used. Another method of transmission error handling
would be to utilise \acrshort{fec} (for example, parity checks),
however this may greatly increase
the complexity of the project, and likely is not necessary given the
frequency rate of errors, packet size, and hardware capabilities.
The Radiohead library supports client-server acknowledgement
with the \lstinline{RHReliableDatagram} class, which allows
checking that a packet has been received correctly, and if not, retransmit.
This can be additionally used to notify when the tracker is no longer
able to transmit to the server (as it would fail to acknowledge the
server's checks). 

The \lstinline{RHDatagram} superclass additionally allows for addressing, meaning 
nodes (receivers) are numbered with an address 0 to 255 and only packets 
addressed to that specific node are accepted, thereby filtering out packets from 
`other' transmitters. This meets the deduplication specification point\footnote{
    \Cref{sec:dedupe}.
}, however suffers the obvious drawback of being limited by address pool. 
An alternative method would be to encode addresses into the packet information manually
rather than relying on a provided class, thereby allowing the address pool criteria 
to be virtually any size. Whatever method is chosen, the transmitters will 
also be able to encode a unique address, allowing an owner to track multiple pets. 

\paragraph{Charging time}
While the transmitter may boast an impressive battery life, the charging
time leaves much to be desired. Protocols like Qualcomm's Quick Charge
may mitigate this, however it is limited to Qualcomm \acrshortpl{soc}.
This works by negotiating a higher current over the \acrshort{usb} connection,
as the default is limited to \qty{500}{\mAh} \cite{axelson:usb}.

\subsection{Quality improvements}
\paragraph{Display}
The Pi \gls{lora} bonnet comes with an \acrshort{oled} display that
was completely unutilised. This could be used to display warnings to the user
to indicate when \acrshort{gps} is lost or the battery levels are low.

\paragraph{Advanced power management}
The Feather M0 supports underclocking, meaning that battery life
could be greatly improved. The \gls{lora} modem also supports commands
to sleep, as otherwise it stays in a passive `listening' mode that draws
some power (around \qty{2}{\mA} \cite{adafruit:loram0}).
Since it is only required to transmit, the radio can be
turned off to save power in the meantime.

\section{Additional remarks}
Some aspects have not been discussed in any great detail. 
Here, such aspects will be focused upon.

\paragraph{Environmental impact}
The project does not have an explicit environmental impact.
Manufacture of electronics in general is an environmental burden, 
as is safe disposal, and this project certainly falls within the purview
of such elements. Such considerations will be vital should this 
project be taken to manufacture, as safe disposal of electronics
and batteries in particular will be necessary. 

\paragraph{Cost} 
The overall cost of the project is not entirely straightforward 
due to the scattered and chaotic nature of the ordering.
All parts used in the project ordered through 
approved vendors totals to £198.11\footnote{
    Before \acrshort{vat}.
}. This cost does not include some components used (i.e. 
the overall cost of the project is greater), the most 
significant of which is the Pi 3. While this retails for 
£40\footnote{
    Available at 
    \href{https://thepihut.com/products/raspberry-pi-3-model-b?src=raspberrypi}{The PiHut \faExternalLink}
}, it was often seen listed on Amazon or eBay 
with a markup over \qty{120}{\%}.
