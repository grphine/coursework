\section{Abstract}
Many pet owners, particularly cat owners, allow their pets to freely roam the outdoors.
However, it is impossible to determine what the pet is up to, or more pertinently, 
where they are and what they are up to. It is also not unusual 
for pets to go missing, and without some method of tracking location, it is impossible to 
determine where they are until they show up by themselves. Should a pet be lost or stuck, 
returning home becomes increasingly unlikely. 

To counteract this, solutions for location tracking pets exist. Many of these solutions 
encounter one drawback or another, for example reliance on infrastructure or high maintenance costs.
This thesis concerns investigating and developing 
a novel solution to the problem of pet tracking using LoRa radio modulation.

This technique allows for highly efficient radio transmissions, gaining  
large distances with minimal power. In effect, it allows the coverage 
of the roaming range of a cat. The difficulty lies in ensuring the radio 
module can be powered and protected from the elements.

Development of this module concerns first basic function, then testing its remote capabilities 
are suitable, and finally giving it a case, so it can be carried around. 
To follow on this project, this design will be refined with a focus on minimising its size 
by moving to a PCB and harness-attachable casing, and running it under a vigorous suite of tests 
until it has been entirely characterised and is suitable for consumer use. 