\section{Evaluation}
The project was, overall, reasonably successful. 
The fundamental aims were met, and many of the objectives outlined were 
followed through. Many of the advanced aims were not met, however. 
In retrospect, some of these were rather ambitious for the timescale 
available, however are certainly worth revisiting and exploring in 
some depth in future projects built upon this one. 

One of the key takeaways is the importance of planning around expected and unexpected issues. 
While hardware delivery delays certainly held the project up significantly, having a mitigation plan 
in place such that progress in some area, whether that be more advanced design work, or a greater 
depth of research, would have been beneficial. This also assumes that alternative hardware cannot be used
at all, whereas in many cases, it potentially may be. 

One of the greatest outcomes was the successful function of the tracker despite the tight time scales.
With a small amount of refinement weatherproofing the case, the tracker is perfectly sufficient 
for function, even if it is not at the point where it is suitable for consumer use. This project 
can be used as an excellent springboard upon which to explore building a genuinely usable product. 

\subsection{Difficulties}
The overall trend indicated by the assessments performed is that many of the early points for the project 
were met, however later and more advanced points were not. The primary limiting factor to progression 
was time constraints, with the biggest contributor being delivery delays. 
As a product development focused project, there was a limited amount of work possible 
without any hardware to develop on. When hardware did finally arrive, it was incomplete,
leading to haphazard development missing components. Delays in antennae delivery (including the 
receiver antenna in particular), shifted distance testing to the final step performed, and limited 
the possibility of additional tests. This has the knock-on effect leaving an incomplete picture 
of the true capabilities of the tracker, and also shifting every subsequent step (like \acrshort{pcb} design)
to the point where they were no longer viable to tackle

In general, development for this project moved rather quickly, with the bulk of programming to produce 
a minimum viable product complete in the span of a week. Hardware delivery issues were expected, given 
global semiconductor shortages, however the true extent of delays were not adequately accounted for and 
thus led to severe delays in the project itself. 


\subsection{Admittances}
The process of code development progressed extremely smoothly. 
Prior experience with the relevant libraries helped in this regard.
Development could have proven extremely difficult, given the 
unfavourable nature of diagnosing \acrshort{rf} issues. 
This project made use of many more scripts than anticipated, 
most of which were for data handling or graphing than direct functionality. 
The programmatic nature made this all very straightforward to design and use. 



