\section{Literature Review}
\label{sec:litrev}

\subsection{Cat behaviour}

Reviewing cat behaviour will allow for a concrete basis upon which to write a specification 
that the tracker will meet. The foremost questions are how far a tame cat may roam, and for how long.

\subsubsection{Roaming distance}
\label{sec:roam}
Hanmer et al. suggests that the maximum distance of \qty{278}{\m}, 
with a median maximum of \qty{99}{\m} \cite{hanmer:urbanisaton}.
It furthermore elaborates on how this decreases with the level of urbanisation, with urban 
areas seeing the smallest home range (\qty{79}{\m}). 
Barratt and Meek further corroborate the 
general value, with a mean range of \qty{2.54}{\hectare} (\qty{159}{\m} radius) and \qty{2.9}{\hectare} (\qty{170}{\m})
respectively \cite{barratt:home, meek:home}. Notably, these both concern felines in a rural area, 
and therefore the range is expectedly larger. Furthermore, the range can vary greatly on the 
cat's personality, with `wandering' cats having a roaming range (\qty{5.1}{\hectare}) an order of magnitude greater than `sedentary'
cats (\qty{0.4}{\hectare}) \cite{meek:home}. 

Notably, feral and unowned cats have a much greater home range (easily \qty{1.5}{\km}) than the 
average tame cat \cite{bengsen:feral,horn:range,molsher2005home}.
Accommodating both feral and tame cats would greatly increase the scope of the project,
especially considering the roaming distance required to be covered increases by a factor 
of six. Therefore, accommodations for feral cats will not be specifically catered for. 
While the project focus is on tame cats, it is understood that tracking any animal 
or moving object may be applied outside this focus. 

\subsubsection{Time spent outdoors}
As with home ranges, this varies greatly with personality. Interestingly enough, a significant 
contributing factor to this value was the owner characteristics. Clancy et al. discusses the 
relationship between owner and cat, amongst which includes time owners allow their cats outdoors.
\qty{40}{\%} of owned cats had some level of outdoor access, and of this, \qty{88.4}{\%} 
spent less than eight hours outside. \qty{97.1}{\%} were not permitted outdoors at night \cite{clancy2003}.
Zhang et al. saw peak activity for cats between 6:00-10:00 and 17:00-21:00 \cite{zhang2022}. 

Altogether, this suggests a required operating capability of eight hours to cover the 90\textsuperscript{th}
percentile, and that typical outdoor activity likely ranges three to four hours. The latter will be 
the critical minimum operating capability. 

A marked finding by Finka et al. indicates that owners who score highly in neuroticism 
on the \gls{bfi} were associated with lower likelihood of permitting ``\textit{ad libitum} 
access to the outdoors'' to their cat \cite{finka2019}. This suggests that the product may be well suited 
to cater to owners who would like to permit their cat more time outdoors. 

\subsection{LoRa}
\gls{lora} radio modulation is a technique derived from \acrshort{css}, 
a combination of chirp signals and spread-spectrum technology. 
Spread-spectrum is a method by which a radio signal has a wider bandwidth 
due to spreading in the frequency domain, which brings with it 
benefits such as resistance to noise and reduced spectral flux density (making 
it more power efficient). Chirp signals are a method by which the 
signal frequency is increased (up-chirp) or decreased (down-chirp) with time \cite{Loukatos2022}. 
\gls{lora} in particular modulates data with instantaneous changes in the starting
frequency to indicate symbol borders \cite{liando2019}. 
Altogether, this gives \gls{lora} modulation the capability of reliable 
transmission of over \qty{3}{\km} in dense suburban areas \cite{augustin:lora}.

The specifics of how modulation is achieved are not necessarily relevant to this project,
as the focus is on utility. The theoretical breakdown of frequency shift chirp modulation 
used by \gls{lora} is provided by Vangelista \cite{vangelista2017}. The patent 
for the technology is currently held by Semtech \cite{patent:lora}.

\gls{lora} uses license-free radio bands. In the UK, this is covered by 
Directive 2011/829/EU \cite{eu:directive}, transferred over to 
Ofcom IR 2030 \cite{ofcom:licenceexempt}. This will be important 
for when selecting hardware, the radio will need to be able 
to transmit at \qty{868.0}{\MHz}.

\subsection{Alternative products}

\subsubsection{Patents}
Investigating existing patents, there appears to be a device 
that performs a similar function published in China \cite{patent:chinatracker}. However,
it makes no mention on how the tracker operates besides the use of \gls{lora} 
and discusses more the construction of the casing. A similar patent can be found 
\cite{patent:bletracker}, which again primarily discusses the construction of the tracker 
casing. However, from the title it can be deduced that the tracker is Bluetooth-based. 

\subsubsection{Similar products}
\label{sec:litsim}
The problem of pet, or more broadly, animal, tracking is a question that has only gained
greater focus in our ever more well-connected world. Tracking farm animals, for example, 
is already an issue being directly tackled \cite{davcev:lorawan}. Crucially, this leverages 
the \acrshort{lpwan} of \gls{lorawan} specifically, and relies less on \gls{lora} as a 
point-to-point communication tool. This is a sensible approach due to the large number of 
incoming connections, which is less applicable here. 

A review of modern pet tracking advances bases itself on existing mobile network 
infrastructure \cite{sivarman:tracker}.
However, one of the key drawbacks it mentions is difficulty of implementation, especially 
amongst agricultural societies. This emphasises the importance of both suitability away 
from infrastructure,
and simplicity and ease of setup for the consumer. 

This problem is present with products the tracker of this project may compete with. 
One of the foremost such, 
Tractive, requires an ongoing subscription due to the usage of an inbuilt \acrshort{sim} card, 
costing as much as £12 per month in addition to the upfront cost of £44.99 \cite{tractive:price, tractive:cat}. 
With reliance on a third party for this service, 
issues like privacy\footnote{
    Tractive may collect ``pet related data that allows to draw conclusions about the pet owner'', which may 
    be of concern for some people \cite{tractive:privacy}.
} and service availability where the company folds are brought into question. 

The alternative is to forego network reliance, and communicate on a point-to-point basis. This is the 
approach this project will follow. One product, the Tabcat V2, follows this approach. 
While the product capabilities vary depending on environmental conditions, the quoted usable range is 
\qty{152}{\m} in typical conditions \cite{tabcat:tracker}. This does not cover the total possible 
roaming range as discussed in \cref{sec:roam}. Their approach is also specifically limited to one 
tracker registered to one tag. This may be an issue in a multi-cat household as \qty{35}{\%} of 
cat-owning households own multiple 
cat \cite{catsprotection:catsreport}. 

The final style discussed here is that employed by products like Tile and AirTags. 
These are Bluetooth devices (i.e. very short 
range only) and have additional reliance on Tile app and Apple device owners respectively, 
meaning the network is even more restricted \cite{tile}.
This also raises additional security concerns, especially regarding 
unwanted tracking \cite{haskell2021, lovejoy2021}.
