

\section{Introduction}

This thesis records the research, analysis, and development of a \gls{lora} radio based 
cat tracker. \gls{lora} is a radio modulation technique that allows for great transmission 
distances and high resistance to noise and interference.
The purpose of such a tracker is to allow pet owners, particularly cat owners,
to allow their pet to roam freely outdoors, while still providing location reporting. 
This may be particularly useful for owners of adventurous pets who live on high floors in 
apartments or otherwise restrictive dwelling. As such, there will be a greater level of emphasis 
placed on suitability in the average home of a dense, urban environment.  

\subsection{Justification}
Many pet owners allow their pets to roam outdoors. None are perhaps as notable as 
the feline for their free-roaming prowess and sense of independence. 
26\% of UK households own a total of eleven million cats, 
of which 63\% spend time outdoors \cite{catsprotection:catsreport}.

Some of these pets may go missing for long periods of time - in some cases, even years
before returning \cite{bbc:missingcat} - that is, if they return at all. 
The anxious pet owner is catered to by the \$2 billion global pet 
wearables market \cite{researchandmarkets:wearablemarket}, providing them many 
options to alleviate their worries. 

Pet wearables that concern tracking location fall into two broad categories
in how they communicate location data back to the owner: 
network based, and radio frequency based. 
\begin{itemize}
    \item Network - leveraging the availability of an existing network, these solutions can have 
            high fidelity, with quick and accurate updates covering a large area. 
            They suffer where a network connection cannot be made.
    \item \acrshort{rf} - usually forming a point-to-point connection between the user and 
            the wearable. The range is usually limited, however no reliance on a pre-existing 
            network is required, reducing operational costs. 
\end{itemize}

\gls{lora} radio is a novel technology for radio communications
that is beginning to find its niche. Most attention is focused on \gls{lorawan},
a network layer built upon the physical layer provided by \gls{lora}.

\gls{lorawan} and \acrshortpl{lpwan} in general are beginning to proliferate and find common 
use, particularly with the Internet of Things \cite{bardyn2016}. 
However, such networks are still not widely available for easy use, and may 
suffer security issues at their current stage of development \cite{eldefrawy2019}. Furthermore, 
while the radio and network definitions have specifications from the LoRa Alliance,
there is as yet no formal infrastructure that can be relied upon. 

However, \gls{lora} radio by itself as an \acrshort{rf} communication method 
is perfectly capable as a transmission medium 
without the network overlay to rely on. As such, it will 
be the focus technology in the development of a new solution
to the problem of pet tracking. 

\subsection{Aim}
\label{sec:aim}
The ultimate goal is a portable \acrshort{gps} tracker that makes use of \gls{lora} to update location
 information on a periodic basis.
The solution will need to monitor the current location of a moving target for a 
significant portion of the day 
and work within a certain radius of a stationary base. 

The focus is on cats in particular, so care may need to be taken to 
ensure the tracker is ergonomically suited and safe for outdoor use.
Due to the wider possible uses of the tracker, portability may also need to be considered. 

As a product designed for use by the average pet owner, the setup must be suitable for use in a home
and be relatively easy to set up. A solution that works 
`out of the box' is better suited. Location reporting needs to be simple to view so that 
it can be done quickly and easily.  

\begin{itemize}
    \item Fundamentals - minimum requirements to demonstrate the viability 
            of the product.
        \begin{enumerate}
            \item Obtain \acrshort{gps} location.
            \item Transmit location to a receiver via \gls{lora}.
            \item Continue transmission remotely for a significant period of time.
            \item Transmit at a regular interval.
        \end{enumerate}
    \item Improvements - upgrades to the product to improve the user's 
            quality of life when using the product.
        \begin{enumerate}
            \item Ergonomics for owner and pet (fitting and wearing). 
            \item Stability of the system (high uptime).
            \item Installable by an average person.
            \item Location information easily viewable. 
        \end{enumerate}
    \item Production - requirements to ensure the product is suitable 
            for manufacture.
        \begin{enumerate}
            \item General safety and suitability to be worn.
            \item Resistance to damage and weather.
            \item Certification requirements and law compliance.
            \item Scale considerations, especially manufacturing.
        \end{enumerate}
\end{itemize}

\subsection{Objectives}
\label{sec:objectives}
The following objectives list will provide guidance for how development 
will progress, and provide a clear set of criteria against which the success of the 
project can be examined. 

This project will follow the numbered steps, and 
aim to meet each objective specified within. 

\begin{enumerate}
    \item Research cat characteristics, like roaming distances and time spent outdoors.
    \item Research \gls{lora} radio requirements for use, and competitive products.
    \item Design the system on a macroscopic level to inform hardware and software requirements. 
            This may include defining a specification of requirements the tracker must achieve. 
    \item Develop the tracker such that it meets the outlined aims. 
    \item Perform tests to verify the efficacy of the tracker\footnote{
        Live testing using animals will require ethical approval to be obtained in advance.
    }.
    \item Develop a \acrshort{pcb} for the tracker to minimise form-factor, and produce 
            adequate casing for it to be fitted to a pet collar or harness. This includes 
            verifying basic functions again, in addition to suitability for intended use on an animal.
    \item Gain required certifications for the product to be moved to manufacture, and design 
            a manufacturing plan.             
\end{enumerate}