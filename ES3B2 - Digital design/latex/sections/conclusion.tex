\section{Conclusion}

\subsection{Evaluation}
The designed game will be evaluated against the criteria. 
\paragraph{User control}
The user takes control of a reticle to select a colour, which is controllable
by tilting the board in the desired movement direction. The sprite is 
bounded within the selection area. 
\paragraph{Object interactions} 
The game progresses by clearing the screen by progressively selecting colours. 
The grid items need to effectively interact with each other in order to determine 
adjacency and when a grid has been matched to the selected colour. However, 
some visual artefacts may occur. 
The player sprite also interacts with the selection grid underneath to select a 
colour and assumes the characteristics of transparency. 
\paragraph{Info bar}
An info bar is drawn from the graphics controller as a banner at the topmost
hierarchical layer, and includes sprites for the game title and IDs. 
\paragraph{Extra hardware}
The game makes use of the in-built accelerometer, available on the SPI interface,
in order to control player movement. It also displays the game title on the 7-segment display. 
\paragraph{Sprites}
The game makes use of 3 sprites, one for the player, title, and IDs. 
There are some issues with the player sprite appearance, where the sprite is shifted somewhat. 
However, these have been mitigated by adjusting the sprite ROM to negate this effect.

\subsection{Reflection} 

\paragraph{Colour blindness support}
The game in its current form is not supportive of most forms of colour vision deficiency, with the likely exception of 
tritanopia and tritanomoly. Support can be increased by adjusting the colour space used, i.e. using a 
Cividis or Viridis colour space \cite{Nunez2018}, or allowing different colour selections by the player.

Support for achromatopsia may prove difficult while retaining game flow and simplicity. One possible solution is 
to use symbolic elements on cells, and improve the animation between cells to indicate matching. This would, however, 
require a significant redesign of the game logic. 

\subsubsection{Player comments}
During player testing, a notable complaint that cropped up was the difficulty of matching a colour,
with the controls feeling `slippery' and `imprecise'. 
This issue can be broken down into two parts - difficulty in identifying the colour to match, and moving to that colour. 
\paragraph{Identification} With the 64 colour selection, differentiating tones in the same quadrant 
can prove difficult. Colour definition can be improved with a greater display resolution, perhaps by moving 
to a more up-to-date display standard, although this would require the requisite adapter to be available on the board. 
\paragraph{Movement} Movement is altogether both too responsive and not responsive enough. This makes fine adjustments
difficult and overshoots frequent. Overshoots can be improved with the current accelerometer implementation 
by compressing the responsive range to cover half the rotation it does currently. In other words, rather than a complete 
half turn returning 16 (the greatest value), only a quarter turn is needed, with every second value skipped. The 
downside is that fine adjustments will be made more difficult. A complete reimplementation of the accelerometer 
will be needed to return data in finer intervals to combat this. 

\subsubsection{General comments}
The development process for the game was not smooth. Debugging, especially for graphical glitches or 
player interactions, proved particularly difficult to manage. 
This is due to how they cannot be observed easily within test benches. Some issues also 
do not appear to be problematic in test benches, whereas fail to render in actual implementation. 
The final implementation also suffers some logical glitches, where a cell remains unmatched. Overall,
the game functions fairly well otherwise. 